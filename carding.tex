\section{Leads} \label{sec:}

General style:
\begin{itemize}
  \item Reverse attitude (low = \enc, high = \disc)
  \item Standard count (Hi/Lo = even)
  \item Attitude on A or Q (denies A)
  \item If needed unblock on K, else count on K
  \item \nth{1}, \nth{3} or \nth{5} in suit
  \item \nth{4} in NT with honor
  \item \nth{2} in NT without honor
  \item In partner's suit always \nth{1}, \nth{3} or \nth{5}
  \item 10 or 9 is always \nth{1} or \nth{3}
  \item MUD for weak 3-counts
\end{itemize}
\vspace{1em}

\begin{table}[H]
  \centering
  \begin{tabular}{|l|l|l|}
    \hline
    & Lead & In Partner's Suit \\ \hline
    Suit & \nth{1}/\nth{3}/\nth{5} & \nth{1}/\nth{3}/\nth{5}\\ \hline
    Notrump & \nth{4} (\nth{2} without honor)& \nth{1}/\nth{3}/\nth{5} \\ \hline
    Subsequent & \nth{1}/\nth{3}/\nth{5} & \nth{1}/\nth{3}/\nth{5}\\ \hline
  \end{tabular}
  \caption{Opening leads style.}
\end{table}

\begin{table}[H]
  \centering
  \begin{tabular}{|l|l|l|}
    \hline
    Card & Combination & Signal \\
    \hline
    Ace   & AK(+), Ax                  & \att                          \\ \grayline
    King  & AK(+), KQ(+)               & If needed unblock, else count \\ \grayline
    Queen & KQ(+), QJ(+), Qx           & \att                          \\ \grayline
    Jack  & J10(+), Jx                 & Count                         \\ \grayline
    10    & HJ10(+), 109(+), 10x       & Count                         \\ \grayline
    9     & H109(+), 98(+), 9x         & Count                         \\ \grayline
    Hi-X  & Xx, xXx                    & Count                         \\ \grayline
    Lo-X  & xxXx, HxX(x), xxxxX, HxxxX & Count                         \\ \hline
  \end{tabular}
  \caption{Leads vs Suit.}
\end{table}

\begin{table}[H]
  \centering
  \begin{tabular}{|l|l|l|}
    \hline
    Card & Combination & Signal \\
    \hline
    Ace   & AK(+), Ax             & \att                          \\ \grayline
    King  & AK(+), KQ(+)          & If needed unblock, else count \\ \grayline
    Queen & KQ(+), QJ(+), Qx      & \att                          \\ \grayline
    Jack  & J10(+), Jx            & Count                         \\ \grayline
    10    & HJ10(+), 109(+), 10x  & Count                         \\ \grayline
    9     & H109(+), 98(+), 9x    & Count                         \\ \grayline
    Hi-X  & Xx, xXx, xXxx(+)      & Count                         \\ \grayline
    Lo-X  & HxxX(+), HHxX(+), HxX & Count                         \\ \hline
  \end{tabular}
  \caption{Leads vs NT.}
\end{table}

\begin{table}[H]
  \centering
  \begin{tabular}{|l|l|l|l|}
    \hline
    & Partner's Lead & Declarer's Lead & Discarding \\ \hline
    1      & Lo = \enc    & Hi/Lo = Even & odd=\enc, even=S/P \\ \grayline
    2 Suit & Hi/Lo = Even &              &                    \\ \grayline
    3      & S/P          &              &                    \\ \hline
    1      & Lo = \enc    & Hi/Lo = Even & S/P                \\ \grayline
    2 NT   & Hi/Lo = Even &              &                    \\ \grayline
    3      & S/P          &              &                    \\ \hline
  \end{tabular}
  \caption{Signals in order of priority.}
\end{table}

For suit preference (S/P) a high card suggests the higher suit and a low card the lower suit.


\section{Discards} \label{sec:}

\begin{itemize}
  \item Italian (aka. Odd-Even) in suit. This means odd=\enc, even Hi/Lo.
  \item Lavinthal in NT. This means Hi/Lo.
\end{itemize}
%%% Local Variables:
%%% mode: latex
%%% TeX-master: "lak"
%%% End:
