\documentclass[a6paper]{article}

\usepackage[T1]{fontenc}
\usepackage[english]{babel}
\usepackage[dvipsnames]{xcolor}
\usepackage{colortbl}
\usepackage{multirow}
\usepackage{tikz}
\usepackage{microtype}
\usepackage{adjustbox}
\usepackage{makecell}
\usepackage{amsmath}
\usepackage{mathtools}
\usepackage[super]{nth}
\usepackage{microtype}
\usepackage{parskip}
\usepackage{multicol}
\usepackage{tabularx}

\usepackage{hyperref}% Should be usually the last package
\usepackage[margin=0.5cm]{geometry}% Should be after hyperref

\newenvironment{bidtable}{%
  \begin{tabular}{l l}%
    }{%
  \end{tabular}%
}

\newcommand\followups[1]{
  \multicolumn{2}{l}{
    \makecell[l]{
      \hspace{2mm}
      \begin{tabular}{l l}
        #1
      \end{tabular}
    }
  }
}

\newcommand{\grayline}{\arrayrulecolor{lightgray}\hline\arrayrulecolor{black}}

% \newcommand\C{\ensuremath{\clubsuit}}
% \newcommand\D{\ensuremath{\diamondsuit}}
% \renewcommand\H{\ensuremath{\heartsuit}}
% \renewcommand\S{\ensuremath{\spadesuit}}

\DeclareSymbolFont{extraup}{U}{zavm}{m}{n}
\DeclareMathSymbol{\varheart}{\mathalpha}{extraup}{86}
\DeclareMathSymbol{\vardiamond}{\mathalpha}{extraup}{87}
\DeclareMathSymbol{\varclub}{\mathalpha}{extraup}{84}
\DeclareMathSymbol{\varspade}{\mathalpha}{extraup}{81}

\newcommand{\C}{\texorpdfstring{\textcolor{ForestGreen}{\raisebox{-0.017em}{\ensuremath{\varclub}}}}{C}}
\newcommand{\D}{\texorpdfstring{\textcolor{YellowOrange}{\raisebox{-0.35pt}{\ensuremath{\vardiamond}}}}{D}}
\renewcommand{\H}{\texorpdfstring{\textcolor{Red}{\raisebox{-0.06em}{\ensuremath{\varheart}}}}{H}}
\renewcommand{\S}{\texorpdfstring{\raisebox{-0.03em}{\ensuremath{\varspade}}}{S}}


\newcommand\N{{\footnotesize NT}}
\newcommand{\+}{\textsuperscript{+}}
\renewcommand{\P}{\texorpdfstring{{\footnotesize{PASS}}}{PASS}}
\newcommand{\X}{{\footnotesize{DBL}}}
\newcommand{\XX}{{\footnotesize{RDBL}}}
\newcommand{\m}{m}
\newcommand{\M}{M}
\newcommand{\mm}{mm}
\newcommand{\MM}{MM}
\newcommand{\oM}{oM}
\newcommand{\om}{om}
\newcommand{\att}{\textsc{att}}
\newcommand{\cnt}{\textsc{ct}}
\newcommand{\unblock}{\textsc{ub}}
\newcommand{\enc}{\textsc{enc}}
\newcommand{\disc}{\textsc{disc}}
\newcommand{\mud}{\textsc{mud}}

\newcommand\tick{\ensuremath{\surd}}

\renewcommand{\arraystretch}{1.2}
\setlength{\lineskip}{0pt}
%\setlength{\tabcolsep}{2mm}

\begin{document}

% System declaration
SYSTEMDEKLARATION

% For
för

Lee Ann \underline{Madissoon}
204952

Kaarel \underline{Kivisalu}
206481

% Basic system
Grundsystem: Precisionsklöver

% Number of dots
Antal prickar: 4

% Opening bids that may require special defensive methods
Öppningsbud som kan kräva speciella försvarsmetoder:\\
1\C=16\+\\
1\D=10-15, 2\+\D\\
2\C=10-15, 6\+\C\\
2\D=10-15, short \D


\newpage

\begin{tabularx}{\textwidth}{|l|l|l|l|X|}
\hline
% Dots & Bid & No of cards & Description & Response \\
% Prickar & Bud & Ant kort & Beskrivning & Svar \\
P & Bud & A & B& S\\
\hline
& 1\C & 0 &16\+ &1\D=0-7, 1\H=8-11,<5\S\\
\hline
3 & 1\D &2& 10-15 &\\
\hline
& 1\H &5& 10-15 &\\
\hline
& 1\S &5& 10-15 &\\
\hline
& 1\N && \makecell[l]{14-16 BAL 1:a, 2:a, 3:e NV\\ 15-17 BAL 3:e VUL, 4:e} &\\
\hline
& 2\C &6&10-15&\\
\hline
1 & 2\D &&\makecell[l]{10-15 \\ 4405/4414/3415/4315}&\\
\hline
& 2\H &5&&\\
\hline
& 2\S &5&&\\
\hline
& 2\N && \makecell[l]{19-20 BAL 1:a, 2:a, 3:e NV\\ 20-21 BAL 3:e VUL, 4:e} &\\
\hline
& 3\C\D &6&&\\
\hline
& 3\H\S &6&&\\
\hline
& 3\N &&&\\
\hline
& 4\C\D &7&&\\
\hline
& 4\H\S &7&&\\
\hline
\end{tabularx}


\newpage

% Slam conventions
Slamkonventioner

\begin{tabularx}{\linewidth}{|X|}
\hline
RKCB 1430, Kickback, Last train, Serious 3NT\\
\hline
DOPI, ROPI\\
\hline
\end{tabularx}

% Other conventions
Övriga konventioner

\begin{tabularx}{\linewidth}{|X|}
\hline
Lebensohl, XYZ\\
\hline
\end{tabularx}

% Defensive bids
Försvarsbud

\begin{tabularx}{\linewidth}{|l|X|}
\hline
1NT & 15-17 (14-18) \\
\hline
2NT & \\
\hline
% Simple cue
Enkelt överbud & \\
\hline
% Jump cue
Dubbelt överbud & \\
\hline
% Jump overcall
Hoppinkliv & 2\N= \\
\hline
%Against 1NT
Mot 1NT & \X=penalty, 2\C=54\+\MM, 2\D=6\+\M, 2\M=5\M 4\+\m, 2\N=55\+\mm\\
\hline
%Against strong 1C
Mot stark 1\C & \X=\MM, \N=\mm \\
\hline
%Against multi 2D
Mot multi 2\D & \\
\hline
%Against weak 2DHS
Mot svaga 2\D\H\S & \\
\hline
%Against 3CD
Mot 3\C\D & \\
\hline
%Against 3HS
Mot 3\H\S & \\
\hline
\end{tabularx}

% Other defensive conventions
Övriga försvarskonventioner

\begin{tabularx}{\linewidth}{|X|}
\hline
\\
\hline
\end{tabularx}

\newpage

% Opening leads
Utspel

\begin{tabularx}{\linewidth}{|l|X|}
\hline
% Against suit
Mot färg & 10-12, MUD\\
\hline
% Against NT
Mot sang & 11-regeln, MUD\\
\hline
\end{tabularx}

% Leads (after the opening lead)
Vändor

\begin{tabularx}{\linewidth}{|l|X|}
\hline
% Towards declarer
Genom spelföraren & \\
\hline
% Towards dummy
Genom träkarlen & \\
\hline
% In partner's suit
I partnerns färg & 1-3-5 \\
\hline
\end{tabularx}

% Signals
Markeringar

\begin{tabularx}{\linewidth}{|l|X|}
\hline
% Strength, schneider=low encourages
Styrka & Schneider \\
\hline
% Length, Malmö=low-high even, Omvänd=reverse
Längd & Omvänd Malmö \\
\hline
% Other
Övriga & Lavinthal, Italienska\\
\hline
\end{tabularx}

% Doubles
Dubblingar

\begin{tabularx}{\linewidth}{|X|}
\hline
Support doubles\\
\hline
\end{tabularx}

\end{document}