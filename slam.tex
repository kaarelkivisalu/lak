\section{Slam Bidding}

\subsection{Kickback Roman Keycard Blackwood}

Without Kickback (keycards are Aces and trump K):

\begin{bidtable}
    4\N & RKC \\
    \followups{
      5\C & 1/4 keycards \\
      5\D & 0/3 keycards \\
      5\H & 2 keycard, no trump Q \\
      5\S & 2 keycard, trump Q \\
    }\\
\end{bidtable}

With Kickback ($X$ is trump):

\begin{bidtable}
    4$X+1$ &  KRCK \\
    \followups{
      4$X+2$ &  1/4 keycards \\
      4$X+3$ &  0/3 keycards \\
      4$X+4$ &  2 keycard, no trump Q \\
      4$X+5$ &  2 keycard, trump Q \\
    }\\
\end{bidtable}

Next step queries trump queen (if not known):

\begin{bidtable}
    $Y$ & trump Q ask \\
    \followups{
      $Y+1$ & no trump Q \\
      $Y+2$ & trump Q \\
    }\\
\end{bidtable}

Next step queries \#kings (then \#queens, ...):

\begin{bidtable}
    $Z$ & \#kings ask \\
    \followups{
    $Z+1$ & 0/3 kings \\
    $Z+2$ & 1/4 kings \\
    $Z+3$ & 2 kings \\
    }\\
\end{bidtable}

\subsection{Is it Kickback?}

The answer is that if a 4-level jump bid could be construed as Kickback, and
there is any way to bid that suit naturally and forcing below game, then it is
Kickback.

If there are two suits, then the cheapest ``impossible'' bid is Kickback for the
lower suit and the next ``impossible'' bis is Kickback for the higher suit.

% \subsection{Preemptive Roman Keycard Blackwood} \label{PRKC}

% % https://bridge-tips.co.il/wp-content/uploads/2020/06/aa_poor_man.pdf
% Over our preempts (2\H\S, 3\D\H\S) 4\C\ asks (over 3\C, 4\D asks), then

% \begin{bidtable}
% $S_1$ & 0 keycards \\
% $S_2$ & 1 keycard, no trump queen \\
% $S_3$ & 1 keycard, with trump queen \\
% $S_4$ & 2 keycards, no trump queen \\
% $S_5$ & 2 keycards, with trump queen \\
% \end{bidtable}

\subsection{Serious 3NT and Last Train}

When a major suit is trumps and slam is not yet ruled out, 3NT is not natural, but rather a waiting bid that shows a good hand in context. If you bybass 3NT, you deny a good hand in context.

The bid one under 4 of the major also shows a good hand in context, and nothing about the cue-bid suit.

Example:

1\H---2\C---2\H---3\H---

\begin{bidtable}
    3\S & cue-bid, does not say whether serious or not \\
    3\N & Serious 3NT (13--15 as 1\H was 10--15), denies a spade cue \\
    4\C & cue-bid, non-serious hand (10--12), denies a space cue, min but not the pits \\
    4\D & Last Train, non-serious hand (10--12), no spade or club cue \\
    4\H & the worst hand (10-11), bad distribution, honor location and slam cards \\
\end{bidtable}

%%% Local Variables:
%%% mode: latex
%%% TeX-master: "lak"
%%% End:
